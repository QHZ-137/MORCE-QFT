\documentclass[11pt]{article}
\usepackage{amsmath,amssymb}
\usepackage{hyperref}
\usepackage{geometry}
\geometry{a4paper, margin=1in}

\title{MORCE-QFT: Teoría Cuántico-Topológica de la Conciencia}
\author{Giovanni Juárez Herrera\
\texttt{vannisgiios@gmail.com}
}
\date{Julio 2025}

\begin{document}

\maketitle

\begin{abstract}
Presentamos MORCE-QFT, un modelo teórico que fusiona la física cuántica y la topología para explicar la naturaleza de la conciencia. Este trabajo establece las bases matemáticas y computacionales, acompañado por simulaciones reproducibles alojadas en un repositorio público.
\end{abstract}

\section{Introducción}
La conciencia ha sido un misterio durante siglos. MORCE-QFT propone una aproximación innovadora basada en campos gauge y topología, ampliando el marco de la física cuántica tradicional...

\section{Marco Teórico}
(Explicar conceptos clave, antecedentes y fundamentos matemáticos)

\section{Metodología}
(Descripción de simulaciones, algoritmos y configuración experimental)

\section{Resultados}
(Resumen de resultados, gráficos y análisis)

\section{Conclusiones}
(Reflexiones finales y potenciales líneas futuras de investigación)

\section*{Repositorio}
El código y datos están disponibles en \url{https://github.com/QHZ-137/MORCE-QFT}.

\end{document}
